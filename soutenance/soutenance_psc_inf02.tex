\documentclass[12pt]{beamer}
\usepackage[polytechnique, psc, complexe, unicouleur]{persobeamer}
%nocouleur ou: unicouleur ou : bicouleur
%complexe ou : simple
%voir dans le package pour changer les couleurs

\title{Projet Scientifique collectif INF02}
\subtitle{Soutenance finale}
\author{}
\date{19 mai 2015}

\begin{document}

    \begin{frame}
      \titlepage
    \end{frame}		

    %% sommaire %%		
    \begin{frame}
      \frametitle{Sommaire}
      %\tableofcontents[pausesections]
      \tableofcontents
    \end{frame}

\section{Rappel du projet}


\begin{frame}
 \frametitle{But du projet}
 

\end{frame}


\begin{frame}
 \frametitle{Modules}
 
 
\end{frame}


\begin{frame}
 \frametitle{Outils extérieurs et points techniques}
 
 
\end{frame}

%%je pense qu'on peut franchement réduire l'état de l'art,
%% à un bref rappel.
%ce n'est pas ce qu'ils veulent entendre à une soutenance
\section{État de l'art}

\begin{frame}
 \frametitle{Traitement du langage naturel}
 
 
\end{frame}

\begin{frame}
 \frametitle{\textit{Extractive summarization}}
 
 
\end{frame}

\begin{frame}
 \frametitle{\textit{Abstractive summarization}}
 
 
\end{frame}


\begin{frame}
 \frametitle{Génération du résumé et évaluation}
 
 
\end{frame}

%je vire copycat/bascet, on détaillera mieux
%%sur le RC lui-même

\section{Sources de données}

\subsection{Données textuelles}

\begin{frame}
 \frametitle{Données textuelles}
 
 
\end{frame}

\subsection{Données conceptuelles}

\begin{frame}
 \frametitle{Extraction de concepts}
 
 
\end{frame}

\begin{frame}
 \frametitle{WordNet}
 
 
\end{frame}

\begin{frame}
 \frametitle{Conceptnet5}
 
 
\end{frame}

\begin{frame}
 \frametitle{Freebase}
 
 
\end{frame}



\section{Réseau de concepts}
% Schrotty !

\subsection{Structure}

\begin{frame}
 \frametitle{Objectifs et utilité du réseau}
 
 
\end{frame}

\begin{frame}[allowframebreaks = 0.7]
 \frametitle{Structure du réseau}
 
 
\end{frame}

\begin{frame}
 \frametitle{Propagation d'activation}
 
 
\end{frame}

\subsection{Construction}

\begin{frame}
 \frametitle{Construction du réseau}
 
 
\end{frame}

\begin{frame}
 \frametitle{Exemple obtenu}
 
 \begin{block}{Gagnants en liens sortants}
 human : 41
water : 44
person : 50
someone : 161
something : 192 
 \end{block}

 \begin{block}{Gagnants en liens entrants}
  soccer_player : 1931
soccer_midfielder : 1575
athlete : 2182
person : 3469
Et bien d'autres... (organisation, country...)
 \end{block}

\end{frame}


%%optionnel, et même beaucoup trop chiant
\begin{frame}
 \frametitle{Choix de programmation}
 
 
\end{frame}


\section{TF-IDF et méthodes statistiques}

\subsection{TF-IDF pour le résumé automatique}

\begin{frame}
 \frametitle{Définition de l'indice TF-IDF}
 
 
\end{frame}

\begin{frame}
 \frametitle{Utilisation de TF-IDF}
 
 
\end{frame}

\subsection{TF-IDF pour l'importance conceptuelle}

\begin{frame}
 \frametitle{}
 
 
\end{frame}


\section{Traitement préalable des données}

\subsection{Analyse syntaxique}

\begin{frame}
 \frametitle{Grammaires}
 
 
\end{frame}

\begin{frame}
 \frametitle{État du texte en fin d'analyse}
 
 
\end{frame}

\subsection{Résolution de pronoms}

\begin{frame}[allowframebreaks = 0.7]
 \frametitle{Résolution de pronoms}
 
 
\end{frame}

\section{Traitement du réseau}

\subsection{Workspace}

\begin{frame}
 \frametitle{Workspace}
 
 
\end{frame}

\begin{frame}
 \frametitle{}
 
 
\end{frame}

\subsection{Algorithme final}

\begin{frame}
 \frametitle{Algorithme final}
 
 
\end{frame}

\begin{frame}
 \frametitle{Exemples}
 
 
\end{frame}

\section{Conclusion}

\begin{frame}
 \frametitle{Conclusion}
 
C'était vraiment trop mythe comme PSC.
 
\end{frame}

\end{document}
