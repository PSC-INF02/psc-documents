\documentclass[12pt]{beamer}
\usepackage[polytechnique, psc, complexe, unicouleur]{persobeamer}
%nocouleur ou: unicouleur ou : bicouleur
%complexe ou : simple
%voir dans le package pour changer les couleurs

\title{Projet Scientifique collectif INF02}
\subtitle{Soutenance finale}
\author{}
\date{\today}

\begin{document}

    \begin{frame}
      \titlepage
    \end{frame}		

    %% sommaire %%		
    \begin{frame}
      \frametitle{Sommaire}
      %\tableofcontents[pausesections]
      \tableofcontents
    \end{frame}

\section{Évolution du projet}
\subsection{Avancement général}

\begin{frame}
  \begin{block}{État actuel}
    \begin{itemize}
      \item Traitement syntaxique offert par systran.
      \item Réseau de concepts opérationnel
      \item TF-IDF opérationnel
      \item Workspace en cours de développement
    \end{itemize}
  \end{block}
\end{frame}

\subsection{Réseau de concepts}
% Schrotty !
\begin{frame}
\frametitle{État actuel du réseau}

\begin{itemize}
 \item Environ 7000 articles analysés (30 jours);
 \item Environ 12 000 concepts et 8000 noms récupérés (chiffre stable);
 \item Génération à partir d'un grand nombre de requêtes, de types différents, d'un réseau de :
 \begin{itemize}
  \item 21 000 n\oe uds ;
  \item 60 000 arêtes ;
 \end{itemize}
 \pause
 Où l'on atteint les limites d'une utilisation intensive de \texttt{networkx}
 
\end{itemize}

\end{frame}

\begin{frame}
\frametitle{Utilisation du réseau}

\begin{itemize}
 \item Pour l'instant : sur des exemples précis, des résultats très encourageants : une bonne interconnectivité et le respect du sens commun ;
 \item Utilisation sur des phrases complètes dans les prochains jours.
\end{itemize}

\end{frame}


\begin{frame}
\frametitle{Exemple}

\begin{itemize}
 \item "wayne rooney"
 \item athlete, person, "soccer player"
 \item "sport event", profession, "break record", sprint, "very competitive", "jump high", "play sport"
 \item television, obsolete, entertainment, addictive, watch, "in live room", bedroom, entertain, appliance...
\end{itemize}

%%!!!!!!!!!!!!!!!!!!!!!!!!!!!!!!!
%%ce n'était pas mentionné dans le rapport
%%mais c'est ce que je pense, là, à l'instant t
\begin{block}{Données importantes}
L'ordre dans lequel vont être activés les n\oe uds, le nombre de n\oe uds que l'on va activer.

Nécessite de complexifier notre vision initiale.
\end{block}


\end{frame}


\subsection{Élaboration du workspace}
\begin{frame}
  \begin{block}{Ajustements de direction}
    \begin{itemize}
      \item Une part non négligeable du travail sur le workspace sera faite sans workers, qui seront implémentés plus tard.
      \item Nous nous orientons désormais vers des résumés très synthétiques (une ou deux phrases au plus)
    \end{itemize}
  \end{block}

  \begin{block}{Structure du workspace}
   \begin{itemize}
     \item Projection du texte dans un espace plus abstrait
     \item L'importance d'un concept dans le workspace correspond à l'importance sémantique dans le texte
   \end{itemize} 
  \end{block}
\end{frame}

\section{Prochaines étapes}

\subsection{Workspace}
\begin{frame}
  \begin{block}{Structure}
  \begin{itemize}
    \item Graphe représentant l'information du texte
    \item Deux types de nœuds~:
      \begin{itemize}
        \item Objets
        \item Attributs de objets
      \end{itemize}
    \item Attributs peuvent évoluer au fil du temps
    \begin{itemize}
    	\item Les changements sont représentés par des liens entre les attributs.
    \end{itemize}
  \end{itemize}
  
  Exemple : "Wayne Rooney was appointed captain as the fans turned their backs on the team."
  \end{block}
\end{frame}

\subsection{Importance conceptuelle}
\begin{frame}
  \begin{block}{Objectif~:}
    Rendre compte de l'importance à accorder à la présence d'un concept
  \end{block}

  \begin{block}{Calcul de l'indice}
    \begin{itemize}
      \item Équivalent dans l'espace des concepts de l'indice idf
      \item Calculé par une analyse statistique sur un corpus
    \end{itemize}
  \end{block}
\end{frame}

\section{Synthèse}
% Un petit résumé, ça fait pas de mal !
\subsection{Calendrier de travail}
\begin{frame}
	\begin{block}{Retour sur le plan prévisionnel}
		\begin{itemize}
			\item Grammaire -> Problème résolu par Systran.
			\item Résumés basés sur TF-IDF pour évaluation de notre algorithme -> Terminé
			\item Réseau de concepts -> 80\% terminé, reste à affiner l'importance conceptuelle.
			\item Workspace -> Spécification largement établie, début de l'implémentation avec un léger retard
		\end{itemize}
	\end{block}
\end{frame}

\begin{frame}
	\begin{block}{Plan actuel}
		\begin{itemize}
			\item Implémenter une surcouche de TF-IDF pour développer le réseau de concepts~: deux semaines.
			\item Entraîner le réseau de concepts sur les données obtenues~: une semaine supplémentaire.
			\item Fin de la spécification du workspace et implémentation~: un mois.
			\item Génération de résumés, évaluation~: mars, avril.
		\end{itemize}
	\end{block}
\end{frame}
	
\end{document}
