\documentclass[a4paper,12pt]{article}
\usepackage[utf8]{inputenc}%%seul package à charger : pour éviter les problèmes sordides d'encodage
\usepackage[fiche,psc]{andre}%%bon eh bien sûr...
%%fiche, psc pour faire une fiche
%%rapport, psc pour faire un rapport
%%pour insérer du code : \langage{java} , puis \begin{lstlisting} \end{lstlisting}
%%pour insérer des guillemets : \eg \og

\title{Réunion du 18 mars 2015}
\author{\membres} %%\membres uniquement avec l'option psc
\date{Le 18 mars 2015}

\begin{document}

%\titrelong%%une page complète
\titrecourt %titre plus court, typiquement pour les fiches

\section{Déroulé}

Antonin subit assez gravement la construction du workspace parce qu'il n'est pas sûr de ce qu'il veut ajouter.

Sarah n'est pas sûre de ce qu'elle doit faire.

\section{Recherche d'entités}

Avec pronom : recherche flemmarde, au plus près.
Tu remontes doucement. Matcher le genre pronom/nom.

Information sujet-objet : vient du parsing des fichiers de crego, permet de matcher les pronoms.

Important : crego to json. Puis json to workspace.

Guillaume : crego to json. Plus simple pour le parsing.
Antonin : json to workspace avec Théo, car il subit beaucoup trop. Il faut regrouper les groupes nominaux.

pipeline découplé, c'est un mot qui fait peur (en deux mots).

Sarah fait un wrapper autour du workspace, basé sur Network.

André : reco pronoms. À partir de Crego : phrases, références.
Réimplémenter une technique simple, sauf si c'est trop compliqué.



\end{document}



