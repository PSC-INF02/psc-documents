\documentclass{beamer}
\usepackage{pgfpages}
\usepackage[francais]{babel}
\usepackage[T1]{fontenc}
\usepackage[utf8]{inputenc}
\usepackage{amsmath,amsthm}
\usepackage{amsfonts,amssymb}

%%%%%%%%%%%%%%%%%%%%%%%%%%%%%%%%%%%%%%%%%%%%%%%%%%%%%%%%%%%%%%%%%%%%%%%%%%%%%%%
%THEMES DU DOCUMENT
%%%%%%%%%%%%%%%%%%%%%%%%%%%%%%%%%%%%%%%%%%%%%%%%%%%%%%%%%%%%%%%%%%%%%%%%%%%%%%%

%– th`emes globaux~: albatross, beetle, crane, fly, seagull.
%– th`emes internes~: lily  orchid, rose.
%– th`emes externes~: whale, seahorse, dolphin
%JuanLesPins Malmoe PaloAlto Berlin Boadilla Copenhagen Hannover Goettingen Montpellier Rochester
%Madrid Antibes Singapore Szeged Warsaw, thème par défaut Ilmenau Luebeck AnnArbor CambridgeUS Dresden

%themes de polices~: professionalfonts serif structurebold structureitalicserif structuresmallcapsserif


\usetheme{Warsaw}
%\usecolortheme{dolphin}
\usefonttheme{structurebold}
%\useinnertheme{nom du theme interne}
%\useoutertheme{split}

%%%%%%%%%%%%%%%%%%%%%%%%%%%%%%%%%%%%%%%%%%%%%%%%%%%%%%%%%%%%%%%%%%%%%%%%%%%%%%%%%%%%%%%%
%AUTRES COMMANDES PRÉLIMINAIRES
%%%%%%%%%%%%%%%%%%%%%%%%%%%%%%%%%%%%%%%%%%%%%%%%%%%%%%%%%%%%%%%%%%%%%%%%%%%%%%%%%%%%%%%%%

\AtBeginSection[]{% génère une tableofcontents au début de chaque section
  \begin{frame}
  \frametitle{Sommaire}
	\tableofcontents[sectionstyle = show/shaded,subsectionstyle = show/show/shaded]
  \end{frame}
}

\setbeamertemplate{itemize item}[triangle]
\setbeamertemplate{enumerate item}[square]
\setbeamertemplate{blocks} [rounded] [shadow=true]

\setbeameroption{show notes}%%ou hide notes
%%\setbeameroption{show notes on second screen=right}


% \title{Réunion de lancement}
\title{Automatic summarizer}
\author{}
\institute{École polytechnique}
\date{}

\logo{\includegraphics[height=0.8cm]{logohori.eps}}

%%%%%%%%%%%%%%%%%%%%%%%%%%%%%%%%%%%%%%%%%%%%%%%%%%%%%%%%%%%%%%%%%%%%%%%%%%%%%%%%%%%
%DEBUT DU DOCUMENT
%%%%%%%%%%%%%%%%%%%%%%%%%%%%%%%%%%%%%%%%%%%%%%%%%%%%%%%%%%%%%%%%%%%%%%%%%%%%%%%%%%%

\begin{document}




%%page de titre
\begin{frame}
  \titlepage{}
\end{frame}

%% sommaire %%
% \begin{frame}
% \frametitle{Sommaire}
% \tableofcontents
% \end{frame}


%%%%%%%%%%%%%%%%%%%%%%%%%%%%%%%%%%%%%%%%%%%%%%%%%%%%%%%%%%%%%%%%%%%%%%%%%%%%
%VRAI DEBUT DU DOCUMENT
%%%%%%%%%%%%%%%%%%%%%%%%%%%%%%%%%%%%%%%%%%%%%%%%%%%%%%%%%%%%%%%%%%%%%%%%%%%%

\section{Introduction}

    \begin{frame}%[allowframebreaks=0.8]
    \frametitle{Motives}%cadre général

        \begin{block}{Need for summarization and difficulties}
            \begin{itemize}
                \item Vast amount of textual data
                \item Complex grammar (natural language)
                \item A lot of prerequisites
            \end{itemize}
        \end{block}


        \begin{block}{Two solutions}
            \begin{itemize}
                \item Abstractive summarization
                \item Extractive summmarization
            \end{itemize}
        \end{block}
    \end{frame}

\section{Project structure}
    \subsection{The project}

        \begin{frame}
        \frametitle{The abstractor in six steps}

            \begin{columns}[t]
                \begin{column}{6cm}
                    \begin{enumerate} %éviter la liste
                        \item Syntactic analysis
                        \item Finding entities
                        \item Finding links between entities
                        \item Concept analysis
                        \item Deeper search of links
                        \item Simple text generation
                    \end{enumerate}
                \end{column}
                %à droite, ce que ça donne aux différents stades

                \begin{column}{7cm}
                    \begin{itemize}
                        \item Subject, Verb, Object\ldots{}
                        \item Zinedine Zidane\ldots{}
                        \item Zidane is a football player\ldots{}
                        \item Football is a sport\ldots{}
                        \item Zinedine Zidane is thus a sportsman,
                        \item Subjects-Actions-Objects-Complements
                    \end{itemize}
                \end{column}
            \end{columns}
        \end{frame}

        \begin{frame}
        \frametitle{Where are we now~?}
            \begin{enumerate}
                \item TF-IDF implementation ... Done
                \item Syntactic and linguistic analysis ... Under Way
                \item ``Smart'' analysis ... Running %but/gros du PSC
                \item Generation and study of the summary's quality ... Bonus
            \end{enumerate}
        \end{frame}

\section{TF-idf}
    \begin{frame}
    \frametitle{TF-IDF(Term Frequency-Inverse Document Frequency)}
        \begin{itemize}
          \item Base for the rest of our work
          \item Programmed a little extractive summary script using TF-IDF
          \item Crawling the web to build a database
        \end{itemize}
    \end{frame}

\section{Analyser's structure}%Guillaume
    \subsection{Concept network}
        \begin{frame}
        \frametitle{Concept network}
            \begin{itemize}
                \item \textbf{Graph of concepts} (objects, actions or more abstract ideas) linked with each other~;
                \item Two concepts are close if it is relevant to think of the second after having met the former~;
                \item Represents an \textbf{understanding of the world} for our program~;
                \item Able to learn and change after analyzing texts.
            \end{itemize}
        \end{frame}

    \subsection{Workspace}
        \begin{frame}
        \frametitle{Workspace}
            \begin{itemize}
                \item Blackboard containing instances of concepts already met and of structures being constructed~; 
                \item Represents \textbf{understanding of the text}
                \item When analysis is over, important structures represent important information from the text.
            \end{itemize}
        \end{frame}

    \subsection{Workers}
        \begin{frame}
        \frametitle{Workers}
        Set of methods to extract information from the text and the concept network.
        For instance~:
            \begin{itemize}
                \item Reading the text and parse it;
                \item Identification of references to the same object~;
                \item Activating nodes in the concept network.
            \end{itemize}
        \end{frame}

        \section{Syntactic Analysis}
        \begin{frame}{Syntactic Analysis}
          \begin{itemize}
            \item First try with nltk grammars (cfg)~;

              Accurate but very rigid.
            \item Currently trying probabilistic grammars (pcfg)

              More flexible, but not as accurate.
         \end{itemize} 
       \end{frame} 
\end{document}
